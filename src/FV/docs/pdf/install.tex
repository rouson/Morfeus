\documentclass[a4paper,12pt]{article}% A4
% \documentclass[letter,12pt]{article}% USA Letter

\usepackage[english]{babel}
\usepackage{fancyvrb}
\usepackage{url}

% *** HYPERREF ***
\usepackage[colorlinks,dvipdfm]{hyperref}
% \usepackage[dvipdfm]{hyperref}
%
% Colored version
\hypersetup{
  linkcolor=blue,
  citecolor=blue,
  urlcolor=blue,
  pdfstartpage=1
}
%
%  B/W version
% \hypersetup{
% linkcolor=black,
% citecolor=black,
% urlcolor=black,
% pdfstartpage=1
% }

\author{Stefano
  Toninel \thanks{University of Bologna; e--mail:
    \texttt{stefano.toninel@mail.ing.unibo.it}}\\
  David P. Schmnidt \thanks{University of Massachusetts -- Amherst;
    e--mail: \texttt{schmidt@ecs.umass.edu}}\\
  Salvatore Filippone \thanks{ University of Rom -- Tor Vergata;
    e--mail: \texttt{salvatore.filippone@uniroma2.it}} \\
  Marco Rorro \thanks{ CASPUR; e--mail: \texttt{rorro@caspur.it}}
}

% \date{\today}

\title{
  NEMO version 1.1\\
  Numerical Engine for Multiphysics Operators\\
  \bigskip
  Install Guide
}


\begin{document}

\maketitle
\newpage
\tableofcontents
\newpage

\section{Supported Platforms}
The code has been successfully tested so far on the following
operating systems/platforms:
\begin{itemize}
\item Linux (kernel 2.6.x) on IA--32 and AMD64;
\item Mac Os X 10.4 (Tiger) on PowerPC--G4.
\end{itemize}

In order to build NEMO a standard Fortran 95 compiler is required. The
following compilers are presently supported:
\begin{itemize}
\item GFortran 4.2.0 or later\footnote{\url{http://gcc.gnu.org/fortran/}}
  included in the free GCC compiler suite
  \footnote{\url{http://gcc.gnu.org/}};
\item Intel Fortran Compiler 9.1 for Linux or later
  \footnote{\url{http://www.intel.com/}}.
\end{itemize}


\section{Prerequisites}

Before compiling NEMO one should check that the following tools and
packages are installed and available to use in the building procedure.

\begin{description}
\item [An ISO Fortran 95 compiler] including support for TR15581
    ``Allocatable Extensions'';  GFortran is the \emph{reference
    compiler} for the NEMO project since it is free, open--source and
  available for practically every platform.
\item [BLAS] Basic Linear Algebra Subprograms. A generic
  implementation is available at
  \url{http://www.netlib.org/blas/}. This is easy to install and works
  well for debugging. For a better performance one can choose
  one of the following options:
  \begin{itemize}
  \item [-] \textbf{ATLAS} (Automatically Tuned Linear Algebra
    Software) available at \url{http://www.netlib.org/atlas/};
  \item [-] a processor--dependent implementation like Intel--MKL
  (Math Kernel Library), ACML (AMD Core Math Library), etc.
  \end{itemize}
\item [LAPACK] Linear Algebra PACKage. A generic implementation is
  available at
  \url{http://www.netlib.org/lapack/}. Processor--optimized BLAS
  distributions, such as Intel--MKL and ACML, usually includes also their
  own LAPACK implementation. \footnote{Presently LAPACK routines are
    used only for benchmarking NEMO's own implementation of some
    numerical methods. Its installation is recommended for developers,
    while optional for users.}
\item [MPI] An implementation of the Message Passing Interface standard,
 like {\bf MPICH} available at \url{http://www.mcs.anl.gov/research/projects/mpich2/};
{\bf MVAPICH} available at \url{http://mvapich.cse.ohio-state.edu/} or
{\bf Open MPI} available at \url{http://www.open-mpi.org/}

\item [PSBLAS] 2.4.x Parallel Sparse BLAS. Available at
  \url{http://www.ce.uniroma2.it/psblas/}.
\item [CGNSlib] CFD General Notation System library. Available at
  \url{http://www.cgns.org/}.
\item [ParMETIS] Parallel METIS, a library for domain
  decomposition. Available at
  \url{http://glaros.dtc.umn.edu/gkhome/metis/parmetis/overview}
\end{description}
Install instructions for each one of the previously listed components
are reported in the next sections.

\subsection{GFortran}

GFortran is a part of the GCC suite. Binary files are usually
included in the most recent Linux distributions, such as Suse, Fedora
Core, etc. However, since this open--source project is evolving very
quickly, new bugs are discovered and old ones are fixed very
frequently. Thus it is strongly recommended to use an up--to--date
version of the compiler.

\subsubsection{Binary Downloading}

At the website \url{http://gcc.gnu.org/wiki/GFortran#download} weekly
up--to--date binaries for all platforms supported by NEMO are
available.

\subsubsection{Snapshot Bootstrap}

Alternatively one can download the most recent snapshot directly from
a GCC mirror site\footnote{\url{http://gcc.gnu.org/mirrors.html}} and
compile it by means of a bootstrap procedure. Here is a step--by--step
guide for compiling GFortran on a UNIX--like box. The default install
path is \verb+$HOME/opt/gcc-4.5.0/+.

\begin{enumerate}
\item Install the prerequisites GMP\footnote{\url{http://gmplib.org/}}
  (GNU Multiple Precision Library) version 4.3.2 (or higher) and
  MPFR\footnote{\url{http://www.mpfr.org/}} version 2.4.2 (or
  higher). Usually Linux distributions include up--to--date releases
  of both packages. Otherwise one has to download the respective source
  files, compile, install them and update the library search path.
\item
  \begin{Verbatim}
    $ cd  ~/opt
    $ mkdir gcc-build
    $ cd gcc-build
    $ mkdir obj
  \end{Verbatim}
\item Download from a GCC mirror site a recent release, for instance
      \verb+gcc-4.5.0.tar.bz2+.

  The file must be saved in \verb+$HOME/opt/gcc-build/+.
\item Expand the \verb+.tar.bz2+ archive:
  \begin{Verbatim}
    $ tar xvjf gcc-4.5.0.tar.bz2
  \end{Verbatim}
  a directory named \verb+gcc-4.5.0+ will be created.
\item Configure the bootstrap procedure:
  \begin{Verbatim}
    $ cd obj
    $ ../gcc-4.5.0/configure --prefix=$HOME/opt/gcc-4.5.0/
  \end{Verbatim}
  This generates the ``Makefiles'' needed by the building procedure.
  If the path for the GMP and MPFR objects is not included in the
  default library search path, one has to specify it explicitly by
  adding in the configure step:
  \begin{Verbatim}
    --with-gmp=/path/to/GMP --with-mpfr=/path/to/MPFR
  \end{Verbatim}
\item Run the bootstrap job for building the package:
  \begin{Verbatim}
    $ nohup make bootstrap > make.log &
  \end{Verbatim}
\item Install \verb+gcc+ and \verb+gfortran+:
\begin{verbatim}
$ make install
\end{verbatim}
\item Update the library and executable search paths by adding the
  following strings:
  \begin{Verbatim}
    PATH=$HOME/opt/gcc-4.5.0/bin:$PATH
    [...]
    export PATH
  \end{Verbatim}
  in \verb+~/.bash_profile+ or \verb+~/.bashrc+, and
  \begin{Verbatim}
    LD_LIBRARY_PATH=$HOME/opt/gcc-4.5.0/lib:$LD_LIBRARY_PATH
    [...]
    export LD_LIBRARY_PATH
  \end{Verbatim}
  in \verb+~/.bashrc+. In particular, the second definition is
  mandatory for building MPICH with GFortran.
\end{enumerate}

\subsection{BLAS}

The following steps explain how to compile and install the generic
implementation of BLAS.

\begin{enumerate}
\item
  \begin{Verbatim}
    $ cd ~/LIB
    $ mkdir tmp
  \end{Verbatim}
\item Download \verb+blas.tgz+ from \url{http://www.netlib.org/blas/}
  and copy it to \verb+~/LIB/tmp+.
\item Untar the archive:
  \begin{Verbatim}
    $ cd tmp
    $ tar xvzf blas.tgz
    $ rm blas.tgz
  \end{Verbatim}
\item Compile the source files:
  \begin{Verbatim}
    $ gfortran -O3 -ffast-math -mtune=<CPUtype> -c *.f
  \end{Verbatim}
\item Build the library:
  \begin{Verbatim}
    $ ar curv libblas-gfortran.a *.o
    $ ranlib libblas-gfortran.a
  \end{Verbatim}
\item Install and clean:
  \begin{Verbatim}
    $ mv libblas-gfortran.a ~/LIB
    $ cd ..; rm -r tmp blas.tgz
  \end{Verbatim}
\end{enumerate}

\subsection{LAPACK}

The building procedure for the LAPACK library requires the
specification of a BLAS implementation and consists of the following
steps.

\begin{enumerate}
\item Download \verb+lapack.tgz+ from
  \url{http://www.netlib.org/lapack/} and copy it to \verb+~/LIB+.
\item Untar the archive:
  \begin{Verbatim}
    $ cd ~/LIB
    $ tar xvzf lapack.tgz
    $ cd LAPACK
  \end{Verbatim}
\item Choose the \verb+Make.inc+ for a specific platform; for instance,
  \begin{Verbatim}
    $ cp INSTALL/make.inc.LINUX make.inc
  \end{Verbatim}
\item Edit the \verb+make.inc+ file, setting the following variables:
  \begin{Verbatim}
    FORTRAN  = gfortran
    OPTS     = -O3 -ffast-math -mtune=<CPUtype>
    [...]
    LOADER   = gfortran
    LOADOPTS = $(OPTS)
    [...]
    BLASLIB   = $(HOME)/LIB/libblas-gfortran.a
    LAPACKLIB = liblapack-gfortran.a
  \end{Verbatim}
\item Build the single and double precision objects:
  \begin{Verbatim}
    $ cd SRC
    $ make single double
    $ cd ..
  \end{Verbatim}
\item Install and clean:
  \begin{Verbatim}
    $ mv liblapack-gfortran.a ~/LIB
    $ cd ..
    $ rm -r LAPACK lapack.tgz
  \end{Verbatim}
\end{enumerate}

\subsection{MPICH}

This section contains the instructions for installing MPICH using
\verb+gfortran+ and \verb+gcc+ respectively as Fortran and C
compiler. The default command in the MPICH distribution for starting
remote sessions during parallel jobs would be \verb+rsh+. For a
security--enhanced use one can choose \verb+ssh+ instead. The
following steps refer to the latter option.

\begin{enumerate}
\item Download \verb+mpich-1.2.6.tar.gz+ (or a later version) and
  copy it to \verb+~/LIB/+
\item Untar the archive:
  \begin{Verbatim}
    $ cd ~/LIB
    $ tar xvzf mpich-1.2.6.tar.gz
    $ cd mpich-1.2.6
  \end{Verbatim}
\item Set up the environment for next building procedure:
  \begin{Verbatim}
    $ export FC=gfortran
    $ export F90=gfortran
    $ export RSHCOMMAND=ssh
    $ export F77_GETARGDECL=" "
  \end{Verbatim}
\item Configure and build:
  \begin{Verbatim}
    $ configure --prefix=$HOME/mpich-gfortran
    $ make
  \end{Verbatim}
\item Install and clean:
  \begin{Verbatim}
    $ make install
    $ cd ..
    $ rm -r mpich-1.2.6 mpich-1.2.6.tar.gz
  \end{Verbatim}
\item In order to avoid the password authentication request at the
  login of every ssh session initiated by the \verb+mpirun+
  command, one can generate a private/public key pair.  If you do not already
  have a key pair,
execute the following steps:
  \begin{Verbatim}
    $ cd ~/.ssh
    $ ssh-keygen -t rsa
    [enter no passphrase]
    $ cat id_rsa.pub >> authorized_keys2
  \end{Verbatim}

Note that the permissions of the authorized\_keys2 file should be set to

 \verb+-rw-r--r-- .+
\item Update search path by adding the following line in
  \verb+~/.bash_profile+ or \verb+~/.bashrc+ :
  \begin{Verbatim}
    PATH=$HOME/LIB/mpich-gfortran/bin:$PATH
  \end{Verbatim}
\end{enumerate}

\subsection{PSBLAS}

The PSBLAS library supplies the numerical kernel of NEMO and requires
the installation of the following packages: BLAS and MPI. The
building procedure consists of the following steps.

\begin{enumerate}
\item Download \verb+psblas2.4.x.tgz+ from
  \url{http://www.ce.uniroma2.it/psblas/} and copy it to
  \verb+~/LIB/+.
\item Untar the archive:
  \begin{Verbatim}
    $ cd ~/LIB
    $ tar xvzf psblas2.4.x.tgz
    $ cd psblas2.4.x
  \end{Verbatim}
\item Configure and build:
  \begin{Verbatim}
    $ ./configure --prefix=~LIB/psblas/2.4.0
    $ make
  \end{Verbatim}
Additional packages path can be specified by adding in the configure step, for
instance:
\begin{Verbatim}
     --with-blas=$HOME/LIB/libblas-gfortran.a
\end{Verbatim}
See \verb+./configure --help+ for additional options.
\item Install and clean:
  \begin{Verbatim}
    $ make install
    $ cd ..
    $ rm -r psblas2.4.x psblas2.4.x.tgz
  \end{Verbatim}
\end{enumerate}


\subsection{CGNSlib}

\begin{enumerate}
\item Download \verb+cgnslib_2.5-4.tar.gz+ (or a later version) from
  \url{http://cgns.sourceforge.net/} and copy it to \verb+~/LIB+.
\item Untar the archive:
  \begin{Verbatim}
    $ cd ~/LIB
    $ tar xvzf cgnslib_2.3.tar.gz
    $ mkdir CGNSlib
    $ mkdir CGNSlib/lib
    $ mkdir CGNSlib/include
    $ cd cgnslib_2.5
  \end{Verbatim}
\item Configure:
  \begin{Verbatim}
    $ configure --prefix=$HOME/LIB/CGNSlib
  \end{Verbatim}
\item Compile:
  \begin{Verbatim}
    $ make
  \end{Verbatim}
\item Install and clean:
  \begin{Verbatim}
    $ make install
    $ cd ..
    $ rm -rf cgnslib_2.5 cgnslib_2.5-4.tar.gz
  \end{Verbatim}
\end{enumerate}

\subsection{ParMETIS}

\begin{enumerate}
\item Download \verb+ParMetis-3.1.1tar.gz+ (or a later version) from
  \url{http://glaros.dtc.umn.edu/gkhome/metis/parmetis/overview} and
  copy it to \verb+~/LIB+.
\item Untar the archive:
  \begin{Verbatim}
    $ cd ~/LIB
    $ tar xvzf ParMetis-3.1.tar.gz
    $ cd ParMetis-3.1
  \end{Verbatim}
\item Compile:
  \begin{Verbatim}
    $ make
  \end{Verbatim}
\item Fix a pending bug in the building procedure:
  \begin{Verbatim}
    $ mkdir tmp
    $ cd tmp
    $ ar xv ../libmetis.a
    $ ar curv ../libparmetis.a parmetis.o
    $ cd ..
    $ ranlib libparmetis.a
    $ rm -rf tmp
  \end{Verbatim}
\item Install and clean:
  \begin{Verbatim}
    $ mkdir ~/LIB/ParMetis
    $ mv libparmetis.a libmetis.a ~/LIB/ParMetis
    $ cd ..; rm -rf ParMetis-3.1 ParMetis-3.1.tar.gz
  \end{Verbatim}
\end{enumerate}

\subsection{Special Remarks}

\begin{enumerate}
\item BLAS, LAPACK, MPI and PSBLAS requires
\emph{distinct installations for different compilers}.
\item CGNSlib and PArMETIS can be built just once, by using the
  default compiler \verb+gcc+.
\item Wherever not specified \verb+$HOME/LIB+ is the \emph{default
    library path}.
\end{enumerate}

\section{NEMO}

\subsection{Code Compiling}

After having installed all packages and tools described in the
previous sections, one should be able to compile successfully NEMO,
according to the following steps.

\begin{enumerate}
\item Enter the folder \verb+Nemo/+ containing the software
  distribution. Typing the \verb+ls+ command should return the
  following list of folders and files:
  \begin{Verbatim}
    $ cd Nemo/
    $ ls
    applications  configure.ac  Makefile     nemo-ab-notes.txt
    autogen.sh    docs          Make.inc.in  README
    config        install-sh    missing      src
    configure     LICENSE       mkdir.sh

  \end{Verbatim}
\item Configure and build:
  \begin{Verbatim}
    $ ./configure --prefix=~LIB/Nemo --with-psblas-dir=~LIB/psblas/2.4.0
    $ make
  \end{Verbatim}
Additional packages path can be specified by adding in the configure step, for
instance:
\begin{Verbatim}
     --with-blas=$HOME/LIB/libblas-gfortran.a
     --with-cgns=~LIB/CGNSlib
     --with-parmetis=~LIB/ParMetis
\end{Verbatim}
See \verb+./configure --help+ for additional options.
\item Install
  \begin{Verbatim}
    $ make install
  \end{Verbatim}
\item Install applications:
  \begin{Verbatim}
    $ cd applications
    $ make
  \end{Verbatim}
In order to build all NEMO--based applications included in the
  distribution.
\end{enumerate}

\subsection{Applications running}
The folder \verb+nemo/applications+ contains different executables.
The subfolder \verb+examples+ contains two subfolders: \verb+input+ and
\verb+mesh+: the former provide the input files whereas the latter the mesh.
In order to run an application, for instance, \verb+steady-conduction+:
\begin{enumerate}
\item Set the input file:
\begin{Verbatim}
    $ cd Nemo/applications
    $ cp examples/input/steady_conduction_3d.inp ./nemo.inp
\end{Verbatim}
Edit the file \verb+nemo.inp+ to change some parameters, choose the resolution method and the format of output (DX ot VTK).
\item Run the application
\begin{Verbatim}
    $ mpiexec -np 16 ./steady-conduction
\end{Verbatim}
where \verb+-np+ specify the number of process to use.
\item Visualize the output file:\\
To visualize DX file open \verb+dx+\footnote{\url{http://www.opendx.org/}} and run the appropriate .net file, for instance \verb+temp_3d.net+. \\
To visualize VTK file, run an application like ParaView\footnote{\url{http://www.paraview.org/}} or Visit\footnote{\url{https://wci.llnl.gov/codes/visit/}}, open the corresponding .vtk  file and choose the properties to visualize.
\end{enumerate}

\subsection{Documentation Compiling}

The folder \verb+Nemo/docs/+ contains two documentation resources.
\begin{enumerate}
\item \verb+Nemo/doc/pdf/+ contains the \LaTeX source files for
  building the code documentation in \verb+.pdf+ format. A precompiled
  version of the current install--guide as well as other \verb+.pdf+
  documents should be already available in the \verb+/Nemo/docs/+
  folder.
\item \verb+Nemo/docs/uml/+ contains the uml diagrams which depict the
  object--oriented structure of the code. They can be analyzed and
  modified by using the free tool
  ArgoUML\footnote{\url{http://argouml.tigris.org/}}.
\end{enumerate}


\subsection{Cleaning}
In order to clean the whole distribution and remove all compiling
products (Fortran \verb+.o+ and \verb+.mod+, as well as \LaTeX files),
type in the base directory \verb+Nemo/+:
\begin{Verbatim}
  $ make clean
\end{Verbatim}



\end{document}
